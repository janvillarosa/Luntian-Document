%%%%%%%%%%%%%%%%%%%%%%%%%%%%%%%%%%%%%%%%%%%%%%%%%%%%%%%%%%%%%%%%%%%%%%%%%%%%%%%%%%%%%%%%%%%%%%%%%%%%%%
%
%   Filename    : chapter_7.tex 
%
%   Description : This file will contain your Conclusion and Recommendations.
%                 
%%%%%%%%%%%%%%%%%%%%%%%%%%%%%%%%%%%%%%%%%%%%%%%%%%%%%%%%%%%%%%%%%%%%%%%%%%%%%%%%%%%%%%%%%%%%%%%%%%%%%%


\Section{Conclusion and Recommendations}
\label{sec:conclusionandrecommendations}

\subsection{Conclusion}
\label{sec:conclusion}

Considering the direction and use of technology today, particularly of tablet devices, the Note Assistant application was developed with the objective of developing an application that will utilize touch, pen and voice modalities to facilitate note making and reviewing on a tablet device.

The features developed for the Note Assistant application were divided into five different modules: User Module, Note Management Module, Note Assistant Module, Review Module, and Note Sharing Module. The user module includes features for account creation and account management. The note management module includes features for editing and organizing notebooks and notebooks.  The Note Assistant Module provides annotations tools used to facilitate note taking and note making. This includes features such as marker, line, text, sticky note, highlighter, eraser, media annotation, multi-media mapping tools. The Note Review Module provides tools that facilitate note reviewing such as integrate annotations, display and hide annotations, and visual revision history tools. The last module is the Note Sharing Module. This module provides a feature for sharing notebooks and notes.

A survey was conducted to evaluate the effectiveness of the application. The survey was done in two iterations. In the first iteration users were asked to use the Note Assistant application in one of their courses for two weeks. In the second iteration users were asked to use the Note Assistant application in one of their courses for at least one week. After testing they were asked to accomplish a survey in which they would evaluate all the features of the application, explain how they used the application, and how the application facilitated their methods of taking, making, and reviewing notes. They were also asked if they had any hesitations or constraints in using the software. And to finish they were asked for recommendations on how the application could be further improved.

According to the survey results, the software was used for taking, making, and reviewing notes mostly when concept mapping is involved. Notes were created in the form of ideas. Users used the text tool and pencil/marker tool to write or type notes. Users also used the media annotation tool to include mostly images and audio notes. These notes were later mapped together. It was also used for taking note of necessary things such as announcements or important details. The software facilitated their method of reviewing notes by presenting the ideas in the notes and allowing the testers to relate to those ideas. A user shared that at times it would also be used the same way Microsoft Word or Notepad is used. The software facilitated their method of taking down notes mainly with the sticky notes where they would put concepts. These would then be used while making notes. Their method of making notes was facilitated by the software largely by the concept mapping feature. The concept mapping feature was said to simplify the ideas and notes given by the teacher and is considered one of the favorite features.

According to the survey the users had hesitations in using the software. One hesitation is due to the environment in the Philippines where gadgets have a tendency to attract muggers. Having notes on paper allows one to review them anywhere he or she goes. Another hesitation in using the application was due to the fact that note taking and making is not part of his learning process. Making notes and using the applications was unfamiliar to the tester. A user had hesitations in using the software initially because he was not sure if it was going to be user friendly, but he overcame this hesitation after using it. Another user had a hesitation because he was afraid of getting viruses. A constraint in using the software was the users comfort with the old word processors. This would occasionally make the tester switch from the Note Assistant application to his tablet's Polaris Office for simplicity. Similar to that constraint is a user would sometimes be required by his teacher to use a specific application. One user had a constraint because of the capability of his device which became slower. 

Note Assistant is an application that was developed to facilitate note taking, making, and reviewing. It integrated different tools that were designed after tools used for taking, making, and reviewing printed notes. The digital tools integrated were not novel. They could be found in different applications and software. It is also important to consider what to include and to what extent. Note Assistant is not over loaded with digital tools, however the commonly used digital tools were considered and also how the tools are used. It is how integration and the design of Note Assistant were made that made the application meet its objectives and that is the innovation. 

\subsection{Recommendations}
\label{sec:recommendations}

These are the recommended improvements and additional features for the system based on the survey evaluation forms, findings, and new information about technology gathered throughout the development.

\subsubsection{Improve Account Management}
\label{sec:improveaccountmanagement}

A tester found the account management easy and simple; however, he found logging in every time the application starts redundant. Account Management can be improved by giving users the option to remain logged in even after closing the application. Account Management can also be improved by adding a feature that can allow users to retrieve or reset their password in case they have forgotten it.

\subsubsection{Improve Notebook Icon}
\label{sec:improvenotebookicon}

The Notebook Icon can be improved by making the notebook name more readable. This can be done by changing the font size of the notebook name or by moving the notebook name, elsewhere; e.g. below the icon.

\subsubsection{Improve Text Areas}
\label{sec:improvetextareas}

A tester encountered a problem when dragging the text box to the edge. There were times when the tester could not get the text box back. Text Areas can be improved by adding bounding boxes and also by implementing a way of locating which part of the text box a user can start dragging with.

\subsubsection{Improve Sharing}
\label{sec:improvesharing}

Sharing can be improved by providing more options for sharing. Wi-Fi compatibility and/or integrating the application with cloud based storage are possible options.

\subsubsection{Add Printing Feature}
\label{sec:addprintingfeature}

Adding a printing feature will eliminate one cause of hesitation in using the software. A tester mentioned that carrying a tablet device and reviewing notes on it in public places can attract muggers.  Implementing a printing feature is recommended in order for users not to favor writing notes on paper over using the Note Assistant application for that reason.

\subsubsection{Integrate with Web Browsers and Existing Applications}
\label{sec:integratexistingapplications}

Integrating Note Assistant with Web browsers can improve efficiency by reducing the number of steps needed to retrieve documents, media, and data from the internet and to include these in the notes of a user.
 
Integrating Note Assistant with widely used applications such as Facebook and Twitter will increase the presence of Note Assistant. It will also make notes more accessible to existing users. Integrating Note Assistant with these social networks will increase information exchange and dissemination. It will also increase collaboration among users. Integration of social media can also help in finding information as there is a number of significant people and official institutions like government bodies, schools, and companies utilizing these networks to circulate information.


\subsubsection{Incorporate New Technologies}
\label{sec:incorporatenewtechnologies}

Technology is continuously changing at a very rapid pace. New technologies emerge within a few years and even a few months. Note Assistant can be integrated with technologies similar to the Microsoft Kinect. As mentioned in the related literature regarding Natural User Interface, Microsoft Kinect is among some of the popular systems referred to as natural user interfaces. It makes interaction more direct by allowing users to use spatial gestures and voice commands to control the console. Note Assistant can be escalated from a mobile application to a collaborative virtual 3D room for sharing notes using Kinect. A collaborative virtual 3D room for note sharing will make this process more direct and it will create a virtual environment that promotes user engagement. According to \cite{rosoff:2011}, a person from the film industry said that the traditional game controller was the main barrier to making video gaming into a mainstream activity. This can also be applied to applications for learning and therefore using the Microsoft Kinect can support learning by making sharing virtual notes a mainstream activity.  