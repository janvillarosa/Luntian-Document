%%%%%%%%%%%%%%%%%%%%%%%%%%%%%%%%%%%%%%%%%%%%%%%%%%%%%%%%%%%%%%%%%%%%%%%%%%%%%%%%%%%%%%%%%%%%%%%%%%%%%%
%
%   Filename    : appendix_C.tex
%
%   Description : This file will contain information about your Resource Persons
%                 
%%%%%%%%%%%%%%%%%%%%%%%%%%%%%%%%%%%%%%%%%%%%%%%%%%%%%%%%%%%%%%%%%%%%%%%%%%%%%%%%%%%%%%%%%%%%%%%%%%%%%%
\Section{Consultation with a Domain Expert}
\label{sec:appendixe}

Building an ontology requires a consultation with a domain expert. The proponents of the research interviewed Professor Esperanza Maribel Agoo, a biologist who teaches in De La Salle University. She specializes in plant conservation, ecology, systematics, taxonomy, bioinformatics, and terrestrial ecology. In the interview, distinguishing features that defined the classification of each plant were discussed like the morphological and molecular characteristics of plants. It was mentioned that the roots, stems, leaves, and flowers are known to have medicinal values in some plants. The leaves are essentially the most common part of a plant that have medicinal value because they undergo the special process of photosynthesis. This process enables them to produce a lot of metabolised products which have a tendency to have medicinal properties. Other plant parts like the stem also acquire these medicinal properties because metabolised products pass through them to be able to go to different parts of the plant.

The proponents asked Professor Agoo if  the location of growth of a plant affects its medicinal value and its usage. She explained that the location and the environment of a plant may affect its medicinal value. There are plants that are originally not medicinal but become so because they develop certain mechanisms to be able to adapt and cope with their habitats. Through that process, plants may produce certain compounds that have medicinal properties.



