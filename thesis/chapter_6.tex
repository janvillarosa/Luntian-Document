%%%%%%%%%%%%%%%%%%%%%%%%%%%%%%%%%%%%%%%%%%%%%%%%%%%%%%%%%%%%%%%%%%%%%%%%%%%%%%%%%%%%%%%%%%%%%%%%%%%%%%
%
%   Filename    : chapter_6.tex 
%
%   Description : This file will contain your Results and Observations.
%                 
%%%%%%%%%%%%%%%%%%%%%%%%%%%%%%%%%%%%%%%%%%%%%%%%%%%%%%%%%%%%%%%%%%%%%%%%%%%%%%%%%%%%%%%%%%%%%%%%%%%%%%


\Section{Results and Observations}
\label{sec:resultsandobservations}    %--note: labels help you with hyperlink editing (using your IDE)

\subsection{Methodology}
\label{sec:methodology}

Note Assistant had two evaluation iterations. In the first iteration Note Assistant was evaluated by four senior students from De La Salle University - Manila for two weeks, two out of which were external testers and the other two were internal testers. In the second iteration Note Assistant was evaluated by four external testers. Three of the testers from the second iteration are students from La Salle Green Hills, one freshman and two sophomores. They tested Note Assistant in class for one week. The third tester is senior from De La Salle University – Manila who tested the application for two weeks. Each user was asked to use the Note Assistant application to take down notes, share notes among each other, and provide annotations in one of their courses. Regular discussions among users and developers were conducted during the testing phase to gather feedback and fix bugs.

After testing, each external tester was asked to accomplish a survey form with questions regarding the functionality, usability and applicability of Note Assistant.  

\subsubsection{Survey}

The survey is divided into two parts: feature evaluation and general questions. The feature evaluation part is further divided into five parts in respect to the function modules.

\begin{flushleft}
\textbf{6.1.1.1 Survey Objectives}
\end{flushleft}

\begin{raggedright}
The following points are the objects that need to be achieved after conducting the survey.
\end{raggedright}

\begin{enumerate}
\item To evaluate existing features and enhance as needed
\item To determine if there are missing features and add these as necessary
\item To evaluate if the software facilitates methods of reviewing, taking, and making notes
\item To understand how the software is actually used for reviewing, taking and making notes
\item To determine hesitations and constraints in using the software
\end{enumerate}

\begin{flushleft}
\textbf{6.1.1.2 Feature Evaluation}
\end{flushleft}

\begin{raggedright}
This section of the survey was created to evaluate existing features and to determine if there are missing features. This section is important to discover what needs to be enhanced and added, with due consideration, based on user evaluations.
\end{raggedright}

\begin{figure}[htbp!]
   \centering
   \includegraphics[height=2cm, width=14cm]{PerformanceDefinitionsFirst.eps} 
   \caption{Performance Definitions First Iteration}
   \label{fig:performancedefinitionfirst}
\end{figure}

\begin{figure}[htbp!]
   \centering
   \includegraphics[height=2cm, width=14cm]{PerformanceDefinitionsSecond.eps} 
   \caption{Performance Definitions Second Iteration}
   \label{fig:performancedefinitionsecond}
\end{figure}

\pagebreak

Five performance definitions were defined in the first evaluation iteration for the users to evaluate existing features as shown in Figure \ref{fig:performancedefinitionfirst}. The performance definitions were later changed in the second iteration as shown in Figure \ref{fig:performancedefinitionsecond}. The changes were made with the consideration of the level of the targeted high school evaluators.

\begin{figure}[htbp!]
   \centering
   \includegraphics[height=4cm, width=14cm]{UserModuleEvaluation.eps} 
   \caption{User Module Evaluation}
   \label{fig:usermoduleevaluation}
\end{figure}

Users were asked to evaluate the existing features by checking the appropriate space and commenting where applicable as shown in Figure \ref{fig:usermoduleevaluation}. Figure \ref{fig:usermoduleevaluation} shows the User Module Evaluation section.

The feature evaluation section has four other parts: B. Note Management, C. Note Assistant, D. Review, and E. Note Sharing. These are the other function modules with different features listed under them. The format used for the modules is consistent with the format shown in Figure \ref{fig:usermoduleevaluation}.

\begin{flushleft}
\textbf{6.1.1.3 General Questions}
\end{flushleft}

\begin{raggedright}
This section of the survey was created to determine how the testers used the system.
\end{raggedright}

\begin{figure}[htbp!]
   \centering
   \includegraphics[height=1cm, width=14cm]{Part2Question1.eps} 
   \caption{Part II: Question 1}
   \label{fig:p2q1}
\end{figure}

\pagebreak

Figure \ref{fig:p2q1} shows the first question under the second section which contains general questions. It is important to find out how the software is used for reviewing, taking, and making notes in order to evaluate the usability and applicability of the application.

\begin{figure}[htbp!]
   \centering
   \includegraphics[height=5cm, width=14cm]{Part2Question2.eps} 
   \caption{Part II: Question 2}
   \label{fig:p2q2}
\end{figure}

The second question (shown in Figure \ref{fig:p2q2}) works in conjunction with the first question. After determining how the software is used, it is important to determine if and how the software facilitates the methods of reviewing, taking, and making notes in order to be able to conclude that the objective of the system was met.

\begin{figure}[htbp!]
   \centering
   \includegraphics[height=6cm, width=14cm]{Part2Question3and4.eps} 
   \caption{Part II: Question 3 and 4}
   \label{fig:p2q3n4}
\end{figure}

Questions 3 and 4 (shown in Figure \ref{fig:p2q3n4}) raise the concerns of hesitations and constraints in using the software. It is important to find out the hesitations and constraints as these affect how the users use the application and how the application accordingly facilitates note taking, making, and reviewing.

\begin{figure}[htbp!]
   \centering
   \includegraphics[height=1.5cm, width=14cm]{Part2Question5.eps} 
   \caption{Part II: Question 5}
   \label{fig:p2q5}
\end{figure}

The last question (shown in Figure \ref{fig:p2q5}) allows the testers to provide their feedback and opinion on the system and how it can be further improved. This is a significant part of the survey as it is important to consider the input of the application users.

\subsection{Results and Analysis}
\label{sec:resultsandanalysis}

\subsubsection{Note Assistant Features Usability}

This section discusses the analysis and interpretation of the results of the first part of the survey conducted. There are five charts, each chart represents a function module and its respective features are represented by stacked columns. Stacked columns are divided according to how the testers rated the features according to their usability, in which features can be rated as: Very useful, Useful, Useful with bugs, Not useful, Not useful with bugs for the first iteration and Very useful, Useful, Neutral, Not very useful, Not useful at all for the second iteration.

\begin{figure}[htbp!]
   \centering
   \includegraphics[height=8cm, width=14cm]{UserModuleFirst.eps} 
   \caption{User Module First Iteration}
   \label{fig:usermodulefirst}
\end{figure}

\pagebreak

\begin{figure}[htbp!]
   \centering
   \includegraphics[height=8cm, width=14cm]{UserModuleSecond.eps} 
   \caption{User Module Second Iteration}
   \label{fig:usermodulesecond}
\end{figure}

\begin{flushleft}
First Iteration
\end{flushleft}

\begin{raggedright}
According to the survey, the Account Creation feature was found very useful by both external testers. Account Management feature was found useful by both the testers as well (shown in Figure \ref{fig:usermodulefirst}). 
\end{raggedright}

\begin{flushleft}
Second Iteration
\end{flushleft}

\begin{raggedright}
According to the survey, the Account Creation feature was found very useful by one of the testers and useful by the other three. The Account Management feature was found very useful by three of the testers and neutral by one tester (shown in Figure \ref{fig:usermodulesecond}).
\end{raggedright} 

Account Creation was found very useful or useful because the Note Assistant application requires users to have accounts. One user found Account Management neutral. The reason behind this could be that the feature is not used frequently. According to another user, account management was simple and easy; however, logging in each time the application is started was redundant.

\begin{figure}[htbp!]
   \centering
   \includegraphics[height=8cm, width=14cm]{NoteManagementModuleFirst.eps} 
   \caption{Note Management Module First Iteration}
   \label{fig:notemanagementmodulefirst}
\end{figure}

\pagebreak
\begin{figure}[htbp!]
   \centering
   \includegraphics[height=8cm, width=14cm]{NoteManagementModuleSecond.eps} 
   \caption{Note Management Module Second Iteration}
   \label{fig:notemanagementmodulesecond}
\end{figure}

\begin{flushleft}
First Iteration
\end{flushleft}
 
\begin{raggedright}
According to the survey both testers found the create notebook, edit notebook, and remove notebook features very useful. The feature arrange notebooks was found to be useful by both users, however one encountered bugs while using it. The features add new note, transfer note, and remove notebook were found to be very useful by one user and useful by the other user. The feature merge notes was found to be useful by both users, however one encountered bugs while using it. The view recent notes feature was found to be very useful by one user and useful by the other user (shown in Figure \ref{fig:notemanagementmodulefirst}).
\end{raggedright}

\begin{flushleft}
Second Iteration
\end{flushleft}

\begin{raggedright}
According to the survey the Create Notebook feature was found to be very useful by two testers, useful by one tester, and neutral by one tester. The Edit Notebook feature was found to be very useful by one tester, useful by two testers, and neural by one tester. The Remove Notebook feature was found to be very useful by three testers and neutral by one tester. The Arrange Notebooks feature was found to be very useful by two testers and useful by the other two testers. The Add New note feature was found to be very useful by three testers and useful by one tester. The Transfer Note feature was found to be very useful by one tester, useful by two testers, and neutral by one tester. The Remove Notebook and Merge Notes features were found to be very useful by two testers, useful by one tester, and neutral by one tester. The View Recent Notes feature was found to be very useful by three users and neutral by one user (shown in Figure \ref{fig:notemanagementmodulesecond}).
\end{raggedright}

The features Arrange Notebooks and Add New Note were found to be useful at the very least by all six testers. The Add New Note feature is essential for any user to create a new note. The Arrange Notebooks feature allows users to arrange their notebooks in an organized manner or a way that suites their needs. All the other features were found to be useful at the very least by five out of the six testers.

\begin{figure}[htbp!]
   \centering
   \includegraphics[height=8cm, width=14cm]{NoteAssistantModuleFirst.eps} 
   \caption{Note Assistant Module First Iteration}
   \label{fig:noteassistantmodulefirst}
\end{figure}

\pagebreak
\begin{figure}[htbp!]
   \centering
   \includegraphics[height=8cm, width=14cm]{NoteAssistantModuleSecond.eps} 
   \caption{Note Assistant Module Second Iteration}
   \label{fig:noteassistantmodulesecond}
\end{figure}

\begin{flushleft}
First Iteration
\end{flushleft}

\begin{raggedright}
The Tools Pane feature, where are the tools can be found, and the multi-media mapping features were found to be very useful by both the testers. The concept mapping feature was a favorite feature for both testers. One tester uses the application mainly for making notes for lectures that involve concept mapping. According to this tester creating notes in forms of ideas is easier to look back at than looking at whole sentences. The Marker, Text, Highlighter, Sticky Notes, and Note Templates features were found to be very useful by one user and very useful by one user as well. The Eraser, Hand, and Media Annotation features were found to be useful by both users. The Ruler feature was found to be useful for one user and not useful for the other; and the Undo and Redo feature was found to be very useful for one user and not useful for the other (shown in Figure \ref{fig:noteassistantmodulefirst}). 
\end{raggedright}

\begin{flushleft}
Second Iteration
\end{flushleft}

\begin{raggedright}
The Tools Pane feature was found to be very useful by one user, useful by two testers, and neutral by one tester. The Pencil/Maker, Text, Hand, and Documents Import tools were found to be very useful by two users and useful by the other two testers. The Highlighter and Note Templates tools were found to be very useful by one user and useful by three users. The Ruler tool was found to be useful by two users and neutral by the other two users. The Eraser tool was found to be very useful by two users, useful by one user, and neutral by one user. The Media Annotation tool was found to be very useful by two users, useful by one user, and not very useful by one user (shown in Figure \ref{fig:noteassistantmodulesecond}).
\end{raggedright}

Among all or the features in the Note Assistant Module, only the Media Annotation tool was found to be not very useful by one user as shown in Figure \ref{fig:noteassistantmodulesecond}. How users rate the tools found in the Note Assistant Module would depend on their style of taking and making notes as there are features that would suffice without the others. The user that rated the Media Annotation tool as not very useful rated 8 out of the 12 tools as useful at the very least. It is probable that his style of taking and making notes rely mostly on the 8 tools and not so much on the Media Annotation tool.

\begin{figure}[htbp!]
   \centering
   \includegraphics[height=8cm, width=14cm]{NoteReviewModuleFirst.eps} 
   \caption{Note Review Module First Iteration}
   \label{fig:notereviewmodulefirst}
\end{figure}

\begin{figure}[htbp!]
   \centering
   \includegraphics[height=8cm, width=14cm]{NoteReviewModuleSecond.eps} 
   \caption{Note Review Module Second Iteration}
   \label{fig:notereviewmodulesecond}
\end{figure}

\pagebreak

\begin{flushleft}
First Iteration
\end{flushleft}

\begin{raggedright}
Under the Note Review module, one user didn't find all three features: Integrate Annotations, Display and Hide Annotations, and Visual Revision History useful. On the other hand, the other user found the Integrate Annotations feature useful and the Display and Hide Annotations and Visual Revision History features very useful. The user that did not find any use for all three features is the same user that claimed never to have found use for the Merge Notes and Arrange Notes features under the Note Management module (shown in Figure \ref{fig:noteassistantmodulefirst}). 
\end{raggedright}

\begin{flushleft}
Second Iteration
\end{flushleft}

\begin{raggedright}
As shown in Figure \ref{fig:notereviewmodulesecond} the Integrate Annotations feature was found to be useful by all four users. The Display and Hide Annotations feature was found to be useful by one tester and neutral by the other three users. The Visual Revision History was found to be very useful by one user, useful by two users, and neutral by one user. 
\end{raggedright}

Among all the modules of Note Assistant, the Note Review Module is the one with the least unanimous evaluation among the testers as useful at the very least. The probable reason behind this is that the features found in this module are more useful when there are more notes and when the application has been used for a long time.

\begin{figure}[htbp!]
   \centering
   \includegraphics[height=8cm, width=14cm]{NoteSharingModuleFirst.eps} 
   \caption{Note Sharing Module First Iteration}
   \label{fig:notesharingmodulefirst}
\end{figure}

\pagebreak

\begin{figure}[htbp!]
   \centering
   \includegraphics[height=8cm, width=14cm]{NoteSharingModuleSecond.eps} 
   \caption{Note Sharing Module Second Iteration}
   \label{fig:notesharingmodulesecond}
\end{figure}

\begin{flushleft}
First Iteration
\end{flushleft}

\begin{raggedright}
The Share Notebooks and Notes feature under the Note Sharing module was not used during the testing phase, but the feature was shown to them and it was found to be useful by both testers.
\end{raggedright} 

\begin{flushleft}
Second Iteration
\end{flushleft}

\begin{raggedright}
The Share Notebooks and Notes feature under the Note Sharing module was used by the two sophomores from La Salle Greenhills. One of the two found the feature to be very useful and the other found it to be useful. The remaining two users were not able to use the feature; however they rated it as very useful.
\end{raggedright}

The Share Notebooks and Notes feature has consistently been rated as useful at the very least by the Note Assistant users even though some of the users have never used it. The reason it is rated so is because sharing of notebooks and notes is a usual action done by students and its importance to the users can be recognized.

