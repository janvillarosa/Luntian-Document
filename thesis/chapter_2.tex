%%%%%%%%%%%%%%%%%%%%%%%%%%%%%%%%%%%%%%%%%%%%%%%%%%%%%%%%%%%%%%%%%%%%%%%%%%%%%%%%%%%%%%%%%%%%%%%%%%%%%%
%
%   Filename    : chapter_2.tex 
%
%   Description : This file will contain your Review of Related Literature.
%                 
%%%%%%%%%%%%%%%%%%%%%%%%%%%%%%%%%%%%%%%%%%%%%%%%%%%%%%%%%%%%%%%%%%%%%%%%%%%%%%%%%%%%%%%%%%%%%%%%%%%%%%

\Section{Review of Related Literature}
\label{sec:relatedliterature}

\subsection{Related Studies on Manual Plant Phenotyping Methods}
\label{sec:relatedsystems}

The greenness or chlorophyll concentration of a leaf can provide a basic physiological status of the plant (Hu, Tanaka, & Tanaka, 2013).  There are different methodologies utilized to measure this characteristic in plants. Some methods are destructive while others are nondestructive.

As cited by Netto, Campostrini, de Oliveira and Bressan-Smith (2005), the processes of measuring the chlorophyll concentration in leaves involved the destruction of the plants in order to extract a leaf tissue.  Some of these made use of organic solvents such as acetone (Mackinney, 1941; Bruisna, 1961), dimethylsulfoxide (DMSO) (Hiscox & Israelstam, 1979) methanol, N,N-dimethyl formamide and petroleum ether (Moran & Porath, 1980; Moran, 1982; Lichtenthaler & Wellburn, 1983; Inskeep & Bloom, 1985) in order to separate the pigment from the tissue. 

	The SPAD 502 meter is a portable device used to quantify the chlorophyll content (Percival, Keary & Noviss, 2008). The meter (as cited by Percival et.al) makes readings based on the optical responses of the leaves in response to light exposure and are measured by the light intensity absorbed by the leaf tissue (Kariya, Machida ,& Matsuzaki, 1982).  These readings are in SPAD (Soil Plant Analysis Development) units. 

The plant’s biomass, on the other hand, is important in analysing the growth rate and functional biology of plants. It also serves as a basis for calculating the net primary production of the crops. There are several methods in determining the plant biomass. A common technique in determining the biomass is to measure the plant's over-dried samples by getting the weight of the plant after it is harvested and dried. However, the method is destructive and measuring several plants of the same type must be taken at the same time (Golzarian et al., 2011).

\subsection{Related Studies on Automated Plant Phenotyping Systems}
\label{sec:mobilecomputing}

The core implementation of PHENOPSIS makes use of a robotic arm to record plant characteristics, take images, if necessary, and cycle through the other plants (Agguirezabal, 2005). The data is fed directly to the computers wherein the measurements taken are outputted as Excel files for analysis. Among the data recorded are weight and times of the irrigation cycle, which are used to maintain the automation cycle of the system. In addition, various sensors measuring environmental factors such as humidity, temperature and light exposure are also used to ensure the proper environmental settings. If measurements in the current conditions exceed the acceptable threshold,  the appropriate instruments are turned on to regulate the variable in question. While this system is expensive to deploy because of the various instruments that are used for automation of not only phenotyping, but also regulation of environmental variables, such implementations can still be considered  for the expansion of the Luntian system  in the future.

HTS Bonit, the system developed by LemnaTec for image analysis, utilizes a conveyer belt that aligns a series of plants for a fixed camera to take an individual image of each for phenotyping (Eberius, n.d.). In order to properly quantify the parameters through the use of image processing, a clear, contrasted, solid color is used as the background of the each plant. Some of the parameters include leaf areas, colour, and height. The first step to determining these variables is to distinguish and obtain the actual plant from everything else in the image. This is processed in the LemnaTec software that is bundled with the system. Afterwards, the colours of the plant are defined into ten (10) classes. Leaf colour can be used to determine plant fitness, nutrition and aid in the identification of disease/s which a plant may have. Additionally, a wide range of colours recognized by the system allows a more accurate evaluation of each plant status. Aside from this, an interesting parameter that can be assessed based on the values of leaf length and area is the biomass of the plant, as there is a good correlation between these characteristics. In addition, this system has also been tested on rice plants, which makes it a viable option in determining the biomass of C4 rice.

The paper published by Tsaftaris and Noutsos in 2009 details a setup that is low cost and easy to deploy in nature, making it the most suitable model among all the other researched related systems to Luntian. In order to achieve such characteristics, the system discussed in the paper makes use of digital cameras which are inexpensive in terms of mass purchase and satisfactory in taking the required images. Additionally, each camera  has been set up to utilize an open source firmware called Canon Hack Development Kit. This firmware allows further manipulation of digital camera options which in turn, allow the researchers to tweak the settings to various factors that are present in the system environment. Examples of some manipulated settings are manipulation of the ultra intervalometer for taking time lapse images and utilizing the long exposure intervalometer for taking night time photos.

In order to easily deliver the images to the computers for processing, each camera makes use of Eye-Fi, a two-in-one functionality SD card that allows for storage as well as WiFi connectivity. This specialized SD card provides a few but significant advantages for the system. The main advantage of using Eye-Fi is the elimination of the need for cables. This allows the set-up to be more compact; it would take less space as there are no wires to be used. Additionally, the set-up would also be more easily modifiable and scalable because of the same reason.

Table 2-1 provides a summary of the four phenotyping systems in comparison with each other and Luntian.  The metrics for comparison are chosen to highlight the parameters which Luntian should improve upon. One of the benefits of using Luntian should be the low cost of deployment. Additionally, the measures are used to present the scope and limitations of the system as well.

\subsection{Related Studies on Phenotyping Pipelines}
\label{sec:multitouchtechnology}

HTPheno is an image processing pipeline specifically made for plant phenotyping was designed by Hartmann, Czauderna, Hoffmann, Stein, and Schreiber in 2011. It  is an open source image processing pipeline designed as an ImageJ plugin that employs image analysis algorithms for high-throughput plant phenotyping. HTPheno is not designed to any specific phenotyping setup and as such, it is made to be flexible and highly adaptable to different plant phenotyping setups and environments. However, the research recommends different setups to optimize the processing of plant images. HTPheno can analyze and collect 6 plant phenotypic traits using only the top and side view images of a plant specimen. For the side view image, it can collect the plant height, width and projected shoot area, for the top view, it can collect its x-extent, y-extent, projected shoot area, and plant diameter.
 
The HTPheno pipeline makes use of different image analysis algorithms such as region definition, object segmentation, morphological operation and finally, compilation of the analysis results.
For the software to properly analyze the phenotypic traits of a plant, calibration is done using image segmentation, which partitions an image into different components or segments. Image segmentation is a crucial step in analyzing and collecting the traits for it will determine which segments will be used to precisely analyze the traits. In HTPheno, this is done through color image segmentation with the multidimensional histogram thresholding approach in both Red Green Blue (RGB) and Hue Saturation Value (HSV) color spaces. Segmentation is done on these two color spaces instead of just one because changing light conditions can affect the precision of the segmentation process. Using RGB alone poses a challenge because instability in light conditions can change the color components entirely. Therefore, the HSV color space is added to correct the RGB values. This algorithm is developed into a module of HTPheno called HTPCalib, where the user can define the different segments in a given image by assigning color values to an image. One drawback to the said module is if foreign objects close to the plant have a color similar to it, that object will be segmented into the plant as well, since the module does segmentation using color values.
 
After all the segments in the image have been identified, the object of interest, the plant segment, is extracted in the image. To reduce the drawback of foreign objects being included in the plant segment, Morphological opening is applied in the extracted image. Morphological opening performs erosion, or reducing the segment of interest by eroding its sides, and then dilation, which expands the segment of interest by enlarging or dilating its sides. Morphological opening removes small foreign objects in the plant segment because of erosion, and it smoothes the sides of the plant segment through dilation. This technique results in a plant segment with a lower noise level.

 Finally, the plant segment is analyzed for phenotypic data. The plant segment is transferred back to the original image, which now forms an outline of the plant. From the segment, the software can calculate the plant x-extent, plant y-extent plant diameter, plant width, and plant height. To calculate the plant size, a scale bar of the plant is added on top of the processed image. These phenotypic data is presented in a tabular format, and the said format can be easily exported for data visualization.

Another image processing pipeline for plant phenotyping is detailed in the  research work A growth phenotyping pipeline for Arabidopsis thaliana integrating image analysis and rosette area modeling for robust quantification of genotype effects, by Arvidsson, Perez-Rodriguez and Mueller-Roeber published in the New Phytologist in 2011. The paper discussed an image processing pipeline for gathering phenotypic data for Arabidopsis thaliana. Unlike the previously reviewed image processing pipeline, HTPheno, This system is developed with a phenotyping system based on Lemnatec Scanalyzer HTS. The camera setup that is also used by the pipeline is similar to PHENOPSIS, which uses a camera mounted in a moving robotic arm for imaging. This means that this system is not designed to be highly adaptable to other automated phenotyping systems.
 
For image processing, the system uses the LemnaGrid image capturing and analysis system included in the Lemnatec Scanalyzer HTS, but the image analysis process is customized for gathering the phenotypic data of Arabidopsis thaliana. This decision to use LemnaGrid results in a highly automated process of image analysis, and together with the Lemnatec Scanalyzer, results in a higher degree of accuracy that can only be attained with both optimal hardware and software setups.
 
To analyze and collect phenotypic data from an image, the first step is to convert it from colored to an 8-bit grayscale image. This grayscale image is made with RGB channel weighting, with more weight added in the Green channel. The weighting is done so the resulting image is bright for green-colored areas. Then, image segmentation is performed on the image, using binary mask. The binary mask is created by selecting pixels with a grayscale intensity greater than 130, which means the brighter areas of the image, or areas of the images that are originally greener, are selected and grouped into the binary mask. Then, holes inside this binary mask that are caused by either particles resting on the plant or spots in the leaves are filled. After this, the binary mask is converted into objects, to create the plant segment. To smooth the leaf edges and reconnect some objects of the plant segment that became disconnected during the segmentation process, the objects are dilated and eroded with Morphological operation, using only one pixel per operation. With this process, the plant segment is now one object, and even objects that are outside the pot are included in the segment as long they are connected to the plant. Phenotypic parameters are then calculated for the plant segment, such as plant area, convex hull (which is a simplified geometric outline of the plant), and plant compactness (which is the plant area divided by the area of the convex hull). These parameters are saved into the database. The images generated by the pipeline is shown, and the user can choose to detect and remove incorrectly identified images, as well as make annotations of the image to include additional information.

Table 2-2 provides a summary of the three image processing systems including Luntian.  The systems are compared based on the utilized image too or platform, flexibility with other automated phenotyping setups, removal of foreign objects and noise removing techniques.