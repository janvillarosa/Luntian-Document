%%%%%%%%%%%%%%%%%%%%%%%%%%%%%%%%%%%%%%%%%%%%%%%%%%%%%%%%%%%%%%%%%%%%%%%%%%%%%%%%%%%%%%%%%%%%%%%%%%%%%%
%
%   Filename    : abstract.tex 
%
%   Description : This file will contain your Research Description.
%                 
%%%%%%%%%%%%%%%%%%%%%%%%%%%%%%%%%%%%%%%%%%%%%%%%%%%%%%%%%%%%%%%%%%%%%%%%%%%%%%%%%%%%%%%%%%%%%%%%%%%%%%
\renewcommand{\abstractname}{Abstract}

\begin{abstract}
Plant phenotyping is a vital process that helps farmers and researchers assess the growth, health, and development of a plant. In the Philippines, phenotyping is done manually, with each plant specimen measured and assessed one by one, but this process is laborious, time-consuming, and prone to human-error. Automated phenotyping systems have attempted to address this problem through the use of cameras and image processing, but these systems are costly to deploy and maintain. In order to alleviate this problem, the research aims to develop a high throughput rice phenotyping system that automates the identification of plant greenness and plant biomass of C4 Rice. The proposed system will be constructed based on various image processing tools, existing systems, and screen house setup. This research aims to create a system that provides an efficient way of phenotyping C4 rice by automating the process.


\begin{flushleft}
\begin{tabular}{lp{4.25in}}
\hspace{-0.5em}\textbf{Keywords:}\hspace{0.25em} & Automated Plant Phenotyping, Rice Phenotyping, Image Processing\\
\end{tabular}
\end{flushleft}
\end{abstract}
