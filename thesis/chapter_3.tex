%%%%%%%%%%%%%%%%%%%%%%%%%%%%%%%%%%%%%%%%%%%%%%%%%%%%%%%%%%%%%%%%%%%%%%%%%%%%%%%%%%%%%%%%%%%%%%%%%%%%%%
%
%   Filename    : chapter_3.tex 
%
%   Description : This file will contain your Research Methodology.
%                 
%%%%%%%%%%%%%%%%%%%%%%%%%%%%%%%%%%%%%%%%%%%%%%%%%%%%%%%%%%%%%%%%%%%%%%%%%%%%%%%%%%%%%%%%%%%%%%%%%%%%%%

\lstset{
         basicstyle=\footnotesize\ttfamily, % Standardschrift
         %numbers=left,               % Ort der Zeilennummern
         numberstyle=\tiny,          % Stil der Zeilennummern
         %stepnumber=2,               % Abstand zwischen den Zeilennummern
         numbersep=5pt,              % Abstand der Nummern zum Text
         tabsize=2,                  % Groesse von Tabs
         extendedchars=true,         %
         breaklines=true,            % Zeilen werden Umgebrochen
         keywordstyle=\color{red},
    		frame=b,         
 %        keywordstyle=[1]\textbf,    % Stil der Keywords
 %        keywordstyle=[2]\textbf,    %
 %        keywordstyle=[3]\textbf,    %
 %        keywordstyle=[4]\textbf,   \sqrt{\sqrt{}} %
         stringstyle=\color{white}\ttfamily, % Farbe der String
         showspaces=false,           % Leerzeichen anzeigen ?
         showtabs=false,             % Tabs anzeigen ?
         xleftmargin=17pt,
         framexleftmargin=17pt,
         framexrightmargin=5pt,
         framexbottommargin=4pt,
         %backgroundcolor=\color{lightgray},
         showstringspaces=false      % Leerzeichen in Strings anzeigen ?        
 }
 
\Section{Theoretical Framework}

\subsection{Note Taking}
\label{sec:notetaking}

\subsubsection{Types of Notes and Annotations}

A number of research papers and articles have used the terms notes and annotations interchangeably but the two terms take different definitions. Notes are records of information, usually used to aid memory, synthesize information or for future references. Examples of notes are relevant text or drawings being written down during meetings and presentations. Reminders such as to-do lists are also examples of notes. Annotation is a note or a mark-up that is made while reading a material. Some annotations are highlights, underlines, sketches, and text done in a document.

A study on Academic Note-taking and Annotation Behavior \cite{Cunningham:2005:TNA:1065385.1065477} classified different notes and annotations according to their content and motivations into 8 categories. The first category which is referred to as Reference is a type of note that constitutes sources that are recorded as URLs, names or phrases. The second category is the Reminder such as to-do items. The third one, which is Idea, is a note that represents a concept that is to be explored later on. The fourth category is Paper conversations. Notes that come from passed notes among students during presentations fall under this category. Fifth category is the Negative note which indicates that the presented work ``requires no further consideration.'' Boredom notes, notes that are usually irrelevant to the presentation, are text that are written down just for the sake of doing something during the presentation or appearing professionally during discussion. Doodles, which are similar to Boredom Notes, are made markings of the note-taker to do away with boredom. The last type is Miscellaneous note which is a category of notes that may be ``related to the social aspect of the conference, expense records, and last minute notes on what to say in one's own presentation''

Another way of classifying types of notes was mentioned in a research on improving the usability of note-taking systems \cite{samulviciute:2009}. The classification was based on the setting. One type of notes is the quick notes that are written down which aims to remind oneself. Another type of notes is the one being recorded during lectures and video conferences where information is fast paced. The third type of note is the written information when one reads a material. The last two mentioned note types were the ones covered by the paper.

In a study conducted on the Relationship of Personal and Public Annotations (Marshall Brush, 2004), the researchers classified the annotations of 11 students into three major annotation categories. The first annotation category is the ``anchor-only'' which constitutes underlines, highlights, circles, and margin bar. Second is the ``content-only'' which constitutes notes, markings, and doodles. Last category is ``compound'' which can be sub-categorized into: ``anchor + content'', ``complex anchor'', and ``complex content.'' The first sub-category, ``anchor + content'', is a combination of the first two major categories. The second, ``complex anchor'' is a combination of anchors such as an underline with an embedded circle. The third sub-category, ``complex content'' which is an uncommon annotation category, is a combination of notes, markings and doodles. Annotation categories and types are shown in Table \ref{tab:annotationcategoryandtypes}.

\begin{table}[ht]   %t means place on top, replace with b if you want to place at the bottom
\centering
\caption{Annotation Categories and Their Corresponding Types} \vspace{0.25em}
\begin{tabular}{|>{\centering\arraybackslash}p{1.5in}|>{\centering\arraybackslash}p{1.5in}|} \hline
Annotation Category & Annotation Types \\ \hline
\multirow{4}{*}{Anchor only} & Underline \\
& Highlight \\
& Circle \\
& Margin bar \\ \hline
\multirow{3}{*}{Content-only} & Note \\
& Mark (e.g. *) \\
& Other (doodles) \\ \hline
\multirow{3}{*}{Compound}& Anchor + content \\
& Complex anchor \\
& Complex content \\ \hline

\end{tabular}
\label{tab:annotationcategoryandtypes}
\end{table}

\subsubsection{Note-taking and Note-making}

Note-taking and Note-making are two important academic skills but note-taking is the most popular and more frequently used term. Colin Neville of University of Bradford elaborated the latter in a discussion on Note-Making \cite{neville:2006}. It was said that Note-taking is when the person records what is simply heard or seen in presentations, lectures, or material. On the other hand, note-making was defined as an advance process being done when the person reviewed and synthesized the notes in such a way that the relevant data and information that have been actively selected, arranged and connected could aid in the person's deeper understanding of the subject. Given that note-making is a more active skill, it is considered to have greater potential to become a more creative experience for the person.

The discussion has stated 3 scenarios where note-making can happen. First is when one is synthesizing multiple readings on a subject such as making Venn diagrams. Second is summarization of the connections in a lecture or article, usually via lines, diagrams or charts. The third scenario is when one adds critical comments regarding the material or the lecture.

A popular note-making method is the Cornell Method. This requires the student to make a vertical line dividing the paper ¼ left and ¾ right. All the notes during discussion will be written at the right side while the left side will be used for the key points and questions regarding each of the notes written down on the right side. At the end of the discussion, the student may write a summary or some comments on the bottom part of the notes then one should test ones comprehension by covering the right side and answering the key points and questions on the left side. A template of Cornell Notes is shown in Figure \ref{fig:cornellnotes}.

\pagebreak

\begin{figure}[htbp!]
   \centering        
   \includegraphics[height=10cm, width=8cm]{CornellMethodTemplate.eps}
   \caption{Cornell Method Template \\Key points and questions are written in section 1. \\All of lecture notes are written in section 2. \\The summary and comments after the discussion are written in section 3. }
   \label{fig:cornellnotes}
\end{figure}


\subsubsection{Note-taking and Note-making Methods}

The Note-Making discussion of Colin Neville also discussed three main types of note-taking (Neville, 2007). These are Linear Notes, Visual or Pattern Notes, and Voice Notes.

Linear Notes are those that help in summing up discussions. This type of note can be in forms of sentences, phrases, abbreviations, all that summarizing the main points of discussion. Linear note taking can also be considered as a note-making when one reviews and re-organize notes, connect and synthesize ideas, or add comments and reflections on ideas summarized. Examples of linear note making include photocopying texts and then annotating them, jotting down notes on ready-punch file paper and card record systems. In Lectures aided by a PowerPoint presentation, students usually download and print a copy of the slides and then add their own notes on the margins or the space available in the paper. If no lecture notes are provided by the lecturer, the ``Note and Review'' approach is more appropriate. This approach uses two or three pages of an A4 paper wherein the first pages contains the notes heard from the lecture. The second and consecutive pages contain the summary, key points, and comments of the person. An example of linear notes is shown in Figure \ref{fig:linearnotes}.

\pagebreak

\begin{figure}[htbp!]
   \centering
   \includegraphics[height=9cm, width=12cm]{LinearNotes.eps} 
   \caption{An example of Linear Notes provided by Monash University}
   \label{fig:linearnotes}
\end{figure}

The second type of note-taking, the Visual or Pattern Notes, have different approaches. Usually, visual or pattern notes involve concepts being linked together to show relationships between concepts. An approach to Visual or Patter Notes is through Mind Maps. Mind mapping starts from a concept in the center and then branches out to new ideas through new concepts and lines that connect them. An example of a mind map is shown in Figure \ref{fig:mindmap}. Another approach that is similar to Mind Mapping is Concept Mapping. The main difference between the two is that Mind Map revolves in a single concept while Concept map is a network of multiple concepts. An example of a Concept Map is shown in Figure  \ref{fig:conceptmap}. Fishbone diagrams, also called ``Ishikawa diagrams,'' are Visual or Pattern Notes that is ideal for cause and effect relationships. This involves a long horizontal arrow that has diagonal lines linked to the arrow. An example of a Fishbone diagram is shown in Figure \ref{fig:fishbonediagram}. Other patterns and diagrams such as Flowcharts, Hierarchies, and Cycles are all considered Visual or Pattern Notes.

\pagebreak

\begin{figure}[htbp!]          
   \centering                   
   \includegraphics[height=9cm, width=14cm]{MindMap.eps}  
   \caption{An example of a mind map that roots from Strategy}
   \label{fig:mindmap}
\end{figure}

\begin{figure}[htbp!]          
   \centering                 
   \includegraphics[height=9cm, width=14cm]{ConceptMap.eps} 
   \caption{An example of a concept map provided by Alberto Cañas and Joseph Novak at CMAP}
   \label{fig:conceptmap}
\end{figure}

\begin{figure}[htbp!]   
   \centering           
   \includegraphics[height=9cm, width=13cm]{FishBone.eps} 
   \caption{An example of a fishbone diagram that shows possible causes of interaction failures that can occur in a communicator}
   \label{fig:fishbonediagram}
\end{figure}


\subsubsection{Note-taking Application Development}

Considering the number of existing note-taking and annotation systems available, different studies have recommended designs, features, and implementations for future note-taking applications. These features are not necessarily the standards for these systems but merely suggestions of researchers.

One of the issues with annotating on electronic documents is that electronic documents' contents may ``reflow'' to many different layouts through its lifetime. With this characteristic of electronic documents, freehand annotations should adapt to reflowing of the contents. A paper \cite{Bargeron03reflowingdigital} discussed a solution on reflowing digital ink annotations. First step is to properly group and classify the digital ink annotations. Second is to anchor the annotation to the its ``surrounding logical context'' Third is to render the annotation in a way that the annotation adapts with the layout and retains style and visual meaning. A study \cite{citeulike:162144} on the relationship of personal annotations and public annotations has made implications on designing annotation systems. The authors suggest that system is better to have a support for converting private annotations to public ones. Usual designs of annotation and note-taking systems have a summarize feature wherein all the notes and annotations are simply compiled into one text. The said study proves that this technique is not effective and thus needs to be improved. ``Anchor + content'' annotation category and ``notes'' annotation type are the annotations most likely being used and converted into notes that will be contributed into an online discussion while ``anchor only'' annotation category are most likely being used as basis for students' summaries.

\pagebreak

The study on improving the usability of digital note-taking systems \cite{samulviciute:2009} has suggested a number of pointers in designing such system. One is considering note-taking system approaches wherein one is just mimicking the traditional paper and pen note-taking process and the other is relying on tools of digital note-taking systems. Note-taking speed of users is considered in designing a system. The familiarity of the user with the system affects one note-taking speed that's why the system has to be straightforward and user-friendly. Removing the copy and paste feature was also suggested because this procedure affects negatively on the attention and typing speed of users. Adding a highlighting feature and allowing note-sharing is said to be beneficial for the learning of users. Three of Egle Samuleviciute's ideas on additional features for note-taking systems are the idea drop feature, support of index card approach, and the overview section. The Idea drop feature is a section in the system that allows users to ``drop'' ideas into. These are ideas not fully expressed and to be reorganized into the main notes by the user later on. The overview section is a small area wherein the overall appearance of the note can be seen. The support for index card approach allows users to take down notes specifically for research papers, speech, reviewing and other more.

\subsection{Natural User Interface}
\label{sec:naturaluserinterface}

\subsubsection{Definition and Characteristics}

Natural User Interface has several definitions but this paper will be focusing on the version of Joshua Blake which is ``an interface designed to reuse existing skills for interacting directly with content'' \cite{blake2011natural}. In his book, Blake expounded on every part of the definition.

He explained that natural user interfaces require ``specific planning efforts in advance'' before designing and implementing the interface. Design should support appropriate interactions among the user, the content and the context \cite{blake2011natural}.

These kinds of interfaces also enable users to reuse existing skills, as mentioned in the definition, to operate the system instead of using the mouse and the keyboard which are the primary input devices of graphical user interfaces and command line interfaces. The term ``existing skills'' refer to innate and learned skills. Natural user interfaces enable users interact with computing devices using these skills such as touching, making gestures, and talking \cite{blake2011natural}.

Natural user interfaces also present an interface which uses metaphors from real world experiences instead of the traditional windows, menus and icons. Removing the concept of windows, menus, and icons as the primary interface elements makes interaction more direct. This does not imply an interface having no controls such buttons and other form elements. This only means that these controls are only secondary to the interface and making the direct interaction as the primary interaction method \cite{blake2011natural}.

\subsubsection{Simple and Composite Skills}

Simple skills are learned skills that are important in interacting with a natural user interface. Joshua Blake defined it as learned skills that depend only on innate abilities \cite{blake2011natural}. Innate abilities are human capabilities that naturally mature such as walking, eating, smelling, and other more. Simple skills are developed out of these abilities such as sniffing, touching and tapping. These skills are the skills being considered by NUIs because it is easily learnable, has low cognitive load and can be reused with other tasks \cite{blake2011natural}.

Composite skills, on the other hand, require more effort to be learned, have higher cognitive load, and are specialized for specific tasks. These skills depend on other simple or composite skills. Examples of these skills are clicking with a mouse and navigating through files and folders \cite{blake2011natural}.


\subsubsection{Cognitive Load}

In the previous section, cognitive load was mentioned as one of the criteria for distinguishing simple skills from composite skills. The concept of cognitive load is important in designing a natural user interface because it somehow determines if an interface is natural. Joshua Blake defined this concept as ``the measure of the working memory used while performing a task'' \cite{blake2011natural}.

There are three types of cognitive load to be considered in designing a natural user interface. One is the intrinsic cognitive load which is described as ``the inherent difficulty of the task.'' The second one is extraneous cognitive load which is the load that is caused by the skills needed to execute the interaction. The third one is germane cognitive load which involved processing and understanding the user interface \cite{blake2011natural}.

The three cognitive load types stress the importance of simple skills when applied on natural user interfaces. Simple skills eliminate extraneous cognitive load from performed tasks making these tasks easier compared to tasks that use of composite skills.

In a research \cite{piolat2005cogeffort} on the cognitive load that people experience during note taking, the said skill was considered to be a complex activity because it both involves comprehension (understanding the subject), and production (selecting and writing down the representation) processes at the same time. Another factor that adds up in the cognitive load for note-takers is the time pressure being experienced during note-taking thus, making them do different processes, which are mostly composite skills, to execute note-taking in a fast manner.

\subsubsection{Natural Interaction Guidelines}

Joshua Blake proposed four guidelines in designing interactions to make them natural \cite{blake2011natural}. First is that users interactions should be designed in a way that they make use of existing skills. Skills may differ from person to person because skills are learned and people may have different specializations. This is why it is more advisable to reuse common human skills than domain-specific skills, which are most of the time, composite skills. But both types of skills are still considered to be reused. The second guideline is that the design, as much as possible, should minimize the cognitive load for each interaction through the use of innate abilities and simple skills. Teaching simple skills are recommended when reusing simple skills is not feasible. The third guideline is that the interaction should provide progressive learning for the user, most especially for complicated tasks. The fourth guideline states that the interface design should use interactions that are ``direct, high-frequency, and appropriate to the context.'' There are three methods for an interaction to be direct according to Blake. Being direct has its different degrees. A graphical user interface is more direct than a command line interface while natural user interface is more direct than a graphical interface. First way for a user's action to be direct in a natural interface is that the user's input is supposed to be physically close to the object being interacted with. Second way is that the object or interface should react at the same time as the user's action. Third is that there should be parallel action between the user's action and the object or interface being acted upon. An example of implementing direct interaction is the use of touch gestures in panning, rotating, and scaling images instead of the use of scrolls, tools, dialog boxes, and resize handles. ``High-frequency'' means that having more and quick interactions than a GUI-style interface. ``Appropriate to the context'' would imply reducing tasks or interface elements that are not appropriate at the moment.

\subsubsection{Natural User Interface Design Principles}

Making an interface natural must follow the guidelines in the previous section. Applying the three main guidelines in a natural user interface introduces some samples that must be considered in designing a natural user interface. A concept introduced by Joshua Blake in his book \cite{blake2011natural} is most related to the first guideline and it is called the Concept of Object Permanence. This human innate ability explains that humans naturally expect objects to exist on the locations where they left them. One of the challenges in developing a natural user interface is a file management because most systems usually ask for the filename and the location where it will be saved before it can be closed. In order to execute this task, users use composite skills which are high in cognitive load. Some programs solved the manual saving process through the use of auto-save feature. There are also programs that after opening them, they present the last document accessed by the user. These features lessen the burden of the user to save every time or whenever one closes the program. These features also follow the concept of object permanence because documents naturally don't disappear after being used.

Another interaction problem that graphical user interfaces have is the method of searching and organizing files within a folder hierarchy because these interactions are not natural. Another problem is the non-fluid transition between views or object states. Some objects in these interfaces change in appearance in an instant when their size and orientation are adjusted. These are solved through the use of document shrinking and dragging across the interface. Through these interactions, object permanence is maintained and making the interface natural since the ``fluid transition keeps the user oriented and maintains the suspension of disbelief'' \cite{blake2011natural}. The reusing of basic skills in different situation is suggested in making an interface natural. One concept that humans learn at a very young age is the containment relationship wherein they would know the relationship between objects and containers as they see them. When categorizing items in a system running on a graphical user interface, the system usually present checkboxes, dropdown list of categories, buttons with arrows or icons that can be dragged. These elements require users to use non-basic skills making the interaction less natural. When implemented in a natural user interface, categorization may be executed through drag and drop on a touch screen or a grab-and-drop in a motion tracking system. In the first guideline in making interactions natural, it was stated that skills are learned, therefore, making some skills not existent to other people. In some cases, a task requires a user to use a non-common skill. In such cases, teaching the user this non-common human skill is an option for interface designers. There are ways in teaching a user what they need to know such as video or animated demonstrations. But the best approach is to create learning tasks and apply the third guideline, which is providing progressive learning for the user.

\subsection{Input Modalities}
\label{sec:inputmodalities}

Input Modalities are the ways how a user can interact with the device.

\subsubsection{Touch}

For a touch capable device, once the user touches the screen, the device will enter touch mode. From this point onward, only view for which \textit{isFocusableInTouchMode()} is true will be focusable. Other views that are touchable will not take focus when touched; they will only fire the on click listeners. Any time a user hits other input than touching the screen, the view device will exit touch mode, and find a view to take focus, so that the user may resume interacting with the user interface without touching the screen again. The touch mode simply indicates whether the last user interaction was performed with the touch screen. The \textit{View.isInTouchMode()} method is used to keep track if it is still in touch mode.

\begin{itemize}
\item \textit{View.isInTouchMode()} method
\\\\  Returns whether the device is in touch mode.
\item \textit{View.isFocusableInTouchMode()} method
\\\\Returns whether the view is focusable in touch mode.
\end{itemize}

When the user performs an action qualified as a touch event, including a press, a release, or any movement gesture on the screen, \textit{View.onTouchListener()} is called.

\begin{itemize}
\item \textit{onTouch()} method
\\\\This allows listeners to get a chance to respond before the target view.
\end{itemize}

\subsubsection{Audio}

Another way of input is through the use of user's voice.

\begin{flushleft}
\textbf{3.3.2.1	Android Supported Audio Formats}
\end{flushleft}

\begin{raggedright}
Figure 3.6 shows the media format support built into the Android platform\footnote[2]{http://developer.android.com/guide/appendix/media-formats.html}.
\end{raggedright}

\pagebreak

\begin{figure}[htbp!]   
   \centering           
   \includegraphics[height=8cm, width=14cm]{AudioFormats.eps} 
   \caption{Audio file formats supported by Android OS}
   \label{fig:audiofileformats}
\end{figure}

\begin{flushleft}
\textbf{3.3.2.2 Audio Capture}
\end{flushleft}

\begin{raggedright}
The Android Multimedia Framework includes support for capturing and encoding a variety of common audio formats so that audio can easily be integrated into the application.
\end{raggedright}

\begin{itemize}
\item \textit{MediaRecorder} API
\\\\These are the ways to record an audio\footnote[3]{http://developer.android.com/reference/android/media/MediaRecorder.html}:
\begin{enumerate}
\item Create a new \textit{MediaRecorder} instance by using \textit{android.media.}\\\textit{MediaRecorder}.
\item Set the audio source to be used for recording using \textit{MediaRecorder.}\\\textit{setAudioSource()} on the \textit{MediaRecorder} instance.
\\Use \textit{MediaRecorder.AudioSource.MIC} to define microphone as the audio source.
\item Set the format of the output file produced during recording using \textit{MediaRecorder.}\\\textit{setOutputFormat()} on the \textit{MediaRecorder} instance.
\item Set the path of the output file to be produced using \textit{MediaRecorder.}\\\textit{setOutputFile()} on the \textit{MediaRecorder} instance.
\item Set the audio encoder to be used for recording using \textit{MediaRecorder.}\\\textit{setAudioEncoder()} on the \textit{MediaRecorder} instance.
\item Use \textit{MediaRecorder.prepare()} on the \textit{MediaRecorder} instance to prepare the recorder to begin capturing and encoding data.
\item Use \textit{MediaRecorder.start()} on the \textit{MediaRecorder} instance to begin capturing and encoding data to the file specified with \textit{MediaRecorder.setOutputFile()}.
\item Use \textit{MediaRecorder.stop()} on the \textit{MediaRecorder} instance to stop recording.
\item Use \textit{MediaRecorder.release()} on the \textit{MediaRecorder} instance to release resources associated with it.
\end{enumerate}
\end{itemize}

\begin{flushleft}
\textbf{3.3.2.3 Media Playback}
\end{flushleft}

\begin{flushleft}
The Android multimedia framework includes support for playing variety of common media types so that audio, video, and images can easily be integrated into the applications.
\end{flushleft}

\begin{itemize}
\item Media Player API
\\\\These are examples how to play an audio\footnote[4]{http://developer.android.com/guide/topics/media/mediaplayer.html}:
\begin{itemize}
\item From local resources
\begin{lstlisting}[frame=single, label=lst:notebookxml, caption=Example of how to play an audio from local resources]
MediaPlayer mediaPlayer = MediaPlayer.create(context, R.raw.sound_file_1);
mediaPlayer.start(); // no need to call prepare(); create() does that for you
\end{lstlisting}
\item From internal URI
\begin{lstlisting}[frame=single, label=lst:notebookxml, caption=Example of how to play an audio from internal URI]
Uri myUri = ....; // initialize Uri here
MediaPlayer mediaPlayer = new MediaPlayer();
mediaPlayer.setAudioStreamType(AudioManager.STREAM_MUSIC);
mediaPlayer.setDataSource(getApplicationContext(), myUri);
mediaPlayer.prepare();
mediaPlayer.start();
\end{lstlisting}
\item From external URLs
\begin{lstlisting}[frame=single, label=lst:notebookxml, caption=Example of how to play an audio from external URLs]
String url = "http://........"; // your URL here
MediaPlayer mediaPlayer = new MediaPlayer();
mediaPlayer.setAudioStreamType(AudioManager.STREAM_MUSIC);
mediaPlayer.setDataSource(url);
mediaPlayer.prepare(); // might take long! (for buffering, etc)
mediaPlayer.start();
\end{lstlisting}
\end{itemize}
\end{itemize}

\subsection{Application Development}
\label{sec:applicationdevelopment}

\subsubsection{Android Development Tools}
\begin{itemize}
\item Eclipse
\\\\This is Android's official development platform. There are tools made for Android Development compatible.
\begin{figure}[htbp!]   
   \centering           
   \includegraphics[height=9cm, width=14cm]{Eclipse.eps} 
   \caption{Eclipse IDE}
   \label{fig:eclipseide}
\end{figure}
\item Android Developer Tools (ADT)
\\\\The ADT is a plug-in for Eclipse that provides a suite of tools that are integrated with the Eclipse IDE. It offers access to many features that helps in developing Android applications quickly. It provides GUI access to many of the command line SDK tools as well as a UI design tool for rapid prototyping, designing, and building of the application's user interface.
\item Android Software Development Kit Tools (SDK Tools)
\\\\The Android SDK includes a variety of tools that helps in the development of mobile applications for the Android platform. The SDK tools are required if you are developing Android applications. The most important SDK tools include the Android SDK and AVD Manager (android), the emulator (emulator), and the Dalvik Debug Monitor Server (ddms).
\begin{itemize}
\item Android SDK and AVD Manager (android)
\\\\If you are using Eclipse, the android tool’s features are integrated into ADT, so you should not need to use this tool directly.
\pagebreak
\begin{figure}[htbp!]   
   \centering           
   \includegraphics[height=9cm, width=14cm]{SDKManager.eps} 
   \caption{Android SDK Manager}
   \label{fig:sdkmanager}
\end{figure}
\begin{figure}[htbp!]   
   \centering           
   \includegraphics[height=9cm, width=14cm]{AVDManager.eps} 
   \caption{Android AVD Manager}
   \label{fig:avdmanager}
\end{figure}
\pagebreak
\item Android Emulator (emulator)
\\\\The Android SDK includes a mobile device emulator — a virtual mobile device that runs on a computer. The emulator is used to develop and to test Android applications without using a physical device. When the emulator is running, one can interact with the emulated mobile device just as he would with an actual mobile device, except that the mouse pointer is used to ”touch” the touch screen and can use some keyboard keys to invoke certain keys on the device.
\begin{figure}[htbp!]   
   \centering           
   \includegraphics[height=9cm, width=11cm]{Emulator.eps} 
   \caption{Android Emulator}
   \label{fig:androidemulator}
\end{figure}
\pagebreak
\item Dalvik Debug Monitor Server (ddms)
\\\\Android ships with a debugging tool called the Dalvik Debug Monitor Server (DDMS), which provides port-forwarding services, screen capture on the device, thread and heap information on the device, logcat, process, and radio state information, incoming call and SMS spoofing, location data spoofing, and more.
\begin{figure}[htbp!]   
   \centering           
   \includegraphics[height=9cm, width=14cm]{DDMS.eps} 
   \caption{DDMS in Eclipse}
   \label{fig:ddms}
\end{figure}
\end{itemize}
\end{itemize}

